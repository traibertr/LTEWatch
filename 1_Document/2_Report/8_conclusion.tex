\documentclass[Report.tex]{subfiles}

\begin{document}

\chapter{Conclusion}

Now that the time has come to complete this project, it is finally time to draw conclusions. First, the least that can be said is that this work was an ambitious project that required many varied and complex tasks to be carried out. Producing a complete prototype from almost scratch is a colossal task that requires many decisions to be made in order to give shape to the project. Many different tasks firstly means a wide variety of skills, which makes this project very interesting, but also requires being efficient at each stage of the project to allow everything to be done on time. This project covered both hardware design and software design, but it also required some creativity to find quick and accessible solutions for each question or problem, which made this work fascinating.\\

This project was my first practical implementation of cellular and \textit{GNSS} application. Not being particularly comfortable with the development of this kind of device, I wanted to ensure things as much as possible by adding a lot of back-up plans and alternative solutions to the design of my prototype. This made the hardware design quite complex, which in part delayed the order of the \textit{PCBs} and their late arrival during the project. Because of this, most of the software was developed on the \textit{nRF9160DK} board with evaluation modules. This way of working forced me to design the software in such a way as to make it as easy as possible to export it to the prototype board. This approach worked well, because the software was ported to the prototype board in less than a week. I then had a second week to sort out all design issues, but as I had been extremely foresighted when designing the hardware, there were only few small changes to be made and the prototype board was validated in one week, leaving me only one weak to finish my report before the project submission date.\\

I am finally very satisfied with the result obtained, because I managed to make a functional prototype which allowed to validate nearly all specifications decided at the beginning of the project. The prototype board is therefore a success. However, I am still a little frustrated that the project stops right after the validation of the prototype. This step still only corresponds to the beginning of the realization of the smart-watch, and I would have really appreciated being able to continue the project to create a functional and finished smart-watch.

\end{document}
